%%% Local Variables:
%%% mode: latex
%%% TeX-master: t
%%% End:
%%% TeX-command-extra-options: "-shell-escape"

\documentclass[a4paper,11pt,bibtotoc,xcolor=dvipsnames]{scrartcl}
\usepackage[T1]{fontenc}
\usepackage[utf8x]{inputenc}
\usepackage{lmodern}
\usepackage{algorithm}
\usepackage{fourier}
\usepackage{tikz}
\usetikzlibrary{matrix,arrows,decorations.pathmorphing}
\usepackage{helvet}

\usepackage[english]{babel}
\usepackage[protrusion=true,expansion=true]{microtype}
\usepackage{mathrsfs, amsmath, amsfonts, amsthm}
\usepackage[toc,page]{appendix}
\usepackage{algpseudocode}
\usepackage{subcaption}
\usepackage{url}
\usepackage{units}
\usepackage{textcomp}
\usepackage{xcolor}
\usepackage{minted}
\usepackage{algorithm}
\setlength{\belowcaptionskip}{+20pt}

\definecolor{bggg}{rgb}{0.92,0.90,0.90}
\BeforeBeginEnvironment{minted}{\bigskip\begin{center}\begin{minipage}{0.92\linewidth}}
\AfterEndEnvironment{minted}{\end{minipage}\end{center}\bigskip}

\setminted[c++]{ %
  autogobble=true, % Automatically remove common whitespace
  bgcolor=bggg,
  framesep=2mm,
  fontsize=\footnotesize }
\usemintedstyle{friendly}

\newcommand*{\code}[1]{\mintinline[bgcolor=bggg, fontsize=\footnotesize]{c++}{#1}}

\title{GPU-accelerated training of deep neural networks in TMVA}
\author{Simon Pfreundschuh}
\date{\today}

\begin{document}

%%% Custom macros

\maketitle

In this report a design for an optimized implementation of deep feedfoward
networks in TMVA is outlined. The primary aim of this project is to enable the
training of neural networks on GPUs. This will be achieved by reimplementing
the training algorithm in a generic way, which hides the structure of the underlying
data types and thus allows efficient offloading of the computations to
GPUs. Moreover, this will make the code independent of vendor specific APIs as
well as simplify the porting of the code to other computing architectures.

The aim of this document is thus to introduce a low-level interface
that separates the general training algorithm from the numerical
operations that require device-specific tuning for optimal
performance. In this way the current, general coordination of the training
can remain unchanged and will be performed by the host, which launches
the kernels that perform the actual calculations on the accelerator
device.

\section{The Neural Network Model}

For this implementation we restrict ourselves to feedforward neural networks,
where the activations $\mathbf{u}^l \in \mathbb{R}^{n_l}$ of a layer $l$ are computed
 from the activations of the previous layer using
\begin{align}
  \mathbf{u}^l = f^l\left(\mathbf{W}^l\mathbf{u}^{l-1} + \boldsymbol{\theta}^l \right)
\end{align}
where $\mathbf{W}^l \in \mathbb{R}^{n_{l-1}}\times\mathbb{R}^{n_l}$ and $\boldsymbol{\theta}^l \in \mathbb{R^{n_l}}$ are
 the weights and bias values of layer $l$ and $f^l$ the corresponding
activation function, which we assume here to be a scalar function
$f:\mathbb{R} \to \mathbb{R}$ that is extended to vector or tensor arguments by
element-wise application.

The training set is assumed to consist of $m$ $n$-dimensional vectors
$\mathbf{x} \in \mathbb{R}^n$. We are assuming the network to
consist of an $n_h$ hidden layers. For a given input vector
$\mathbf{x}_i$, the ouput layer of the neural network computes an
ouput vector $\mathbf{u}^{l_h}$, which is transformed into a
probability or a class prediction by applying a suitable
transformation.

\section{Method}

\subsection{Neural Network Training}

Deep neural networks are trained by optimizing a loss function $J\left
( \mathbf{y},\mathbf{u}^{l_h},\mathbf{W} \right )$ that quantifies the
correctness of the network prediction corresponding to the activations
of the output layer $\mathbf{u}^{l_h}$ with respect to the true values
$\mathbf{y}$ and possibly also including regularization terms, which
are functions of the weights $\mathbf{W}$ of the network. This
objective function is then minimized by applying a gradient-based
optimization technique, usually a modification of the gradient descent
method. The key to scalable training of deep neural networks is the
training on small, randomized subsets of the training data, so called
batches. On each batch the gradient is computed and used to train the
network. The general method can thus be written as follows:

\begin{algorithm}
  \caption{Neural Network Training} \label{alg:gn}
\begin{algorithmic}[1]
  \State Initialize weight and bias variables
  \Repeat
  \State Generate random batch of size $b$
  \State Propagate neural values foward through the network
  \State Compute gradients on the batch using backward propagation
  \State Apply minimization step
  \Until{converged}
\end{algorithmic}
\end{algorithm}

Here we will primarily focus on the steps in lines $4,5$ and $6$, which contain the
computationally demanding operations that will be offloaded to the accelerator device.

\subsection{Forward Propagation}

For the forward propagation, we will consider the propagation of the hole batch
through the network, which will expose an additional axis of parallelization.
The training input can then be viewed as as two-dimensional tensor $x_{i,j}$ with
the index $i$ running over the training samples in the batch and the index $j$
running over the features of each training sample.

The neuron values of each layer can then also be represented by
two-dimensional tensors $u^l_{i,j}$, with the index $i$ running over
the training samples in the batch and the index $j$ running over the
neurons in the given layer. We will refer to this tensor as the activation
matrix of the given layer. The forward propagation of the neuron activations through
the network is illustrated in Figure~\ref{fig:forward}. To make the equations more
easily readable Einstein notation is used, meaning that repeated indices
imply a sum over those indices with the exception of indices that appear on the
left-hand side of an equation.

\begin{figure}[!h]
\centering
\resizebox{\linewidth}{!}{%
  \begin{tikzpicture}[>=latex]
% les matrices
\matrix (X) [matrix of math nodes,%
             left delimiter  = {[},%
             right delimiter = {]}] at (0, -1)
{%
  x_{0,0} & \ldots & x_{0,m}  \\
  x_{1,0} & \ldots & x_{1,m}  \\
   \vdots &        & \vdots   \\
  x_{n,0} & \ldots & x_{n,m}  \\
};

%l1
\draw node at (4,2.5) {$u^1_{i,j} = f^1 \left (W^1_{j,k}x_{i,k} + \theta^1_j \right )$};
\draw[fill=black!20, draw=none] (3,2) rectangle (5,-4) {};
\draw node[fill=blue!50, circle] (l1) at (4, 1) {$u^{1}_{i,j}$};
\draw node[fill=green!50, circle] (W1) at (4, -1) {$W^{1}_{i,j}$};
\draw node[fill=green!50, circle] (t1) at (4, -2.5) {$\theta^{1}_{i}$};

\draw [->,thick]%
([xshift=0.4cm]X.east) to%
([xshift=-0.2cm]l1.south west);

\draw [->,thick]%
([yshift=0.1cm]W1.north) to%
node[yshift = 0.2cm, above]{}%
([yshift=0.-0.1cm]l1.south);

\draw [->, thick, out=135, in=225]%
([yshift=0.1cm]t1.north west) to%
node[yshift = 0.2cm, above]{}%
([yshift=0.-0.1cm]l1.south west);


%l2
\draw node at (8,2.5) {$u^2_{i,j} = f^2 \left (W^2_{j,k}u^1_{i,k} + \theta^2_j \right )$};
\draw[fill=black!20, draw=none] (7,2) rectangle (9,-4) {};
\draw node[fill=blue!50, circle] (l2) at (8, 1) {$u^{2}_{i,j}$};
\draw node[fill=green!50, circle] (W2) at (8, -1) {$W^{2}_{i,j}$};
\draw node[fill=green!50, circle] (t2) at (8, -2.5) {$\theta^{2}_{i}$};

\draw [->,thick]%
([xshift=0.2cm]l1. east) to%
([xshift=-0.2cm]l2.west);

\draw [->,thick]%
([yshift=0.1cm]W2.north) to%
node[yshift = 0.2cm, above]{}%
([yshift=0.-0.1cm]l2.south);

\draw [->, thick, out=135, in=225]%
([yshift=0.1cm]t2.north west) to%
node[yshift = 0.2cm, above]{}%
([yshift=0.-0.1cm]l2.south west);

%ld
\draw node[draw=none] (ld) at (10.5, 1) {$\ldots$};
\draw [->,thick]%
([xshift=0.2cm]l2.east) to%
([xshift=-0.2cm]ld.west);

%lnh
\draw node at (13,2.5) {$u^{n_h}_{i,j} = f^{n_h} \left (W^{n_h}_{j,k}u^{n_h-1}_{i,k} + \theta^{n_h}_j \right )$};
\draw[fill=black!20, draw=none] (12,2) rectangle (14,-4) {};
\draw node[fill=blue!50, circle] (lnh) at (13, 1) {$u^{n_h}_{i,j}$};
\draw node[fill=green!50, circle] (Wnh) at (13, -1) {$W^{n_h}_{i,j}$};
\draw node[fill=green!50, circle] (tnh) at (13, -2.5) {$\theta^{n_h}_{j}$};

\draw [->,thick]%
([xshift=0.2cm]ld.east) to%
([xshift=-0.2cm]lnh.west);

\draw [->,thick]%
([yshift=0.1cm]Wnh.north) to%
node[ above]{}%
([yshift=0.-0.1cm]lnh.south);

\draw [->, thick, out=135, in=225]%
(tnh.north west) to%
node[yshift = 0.2cm, above]{}%
(lnh.south west);

% \matrix (Y) [matrix of math nodes,%
%              left delimiter  = {[},%
%              right delimiter = {]}] at (16, -1)
% {%
%   \hat{y}_{0} \\
%   \hat{y}_{1} \\
%   \vdots\\
%   \hat{y}_{n} \\
% };

% \draw [->, thick]%
% ([xshift=0.2cm]lnh.south east) to%
% ([xshift=-0.4cm]Y.west);

% \matrix (Y) [matrix of math nodes,%
%              left delimiter  = {[},%
%              right delimiter = {]}] at (16, -5)
% {%
% y_{0} \\
% y_{1} \\
% \vdots\\
% y_{n} \\
% };

\draw node[fill=red!50, circle] (E) at (17, -1) {$J(\mathbf{y},\mathbf{\hat{y}})$};
% Error function

\draw [->, thick]%
([xshift=0.4cm]lnh.east) to%
([xshift=-0.2cm]E.west);


% \draw [->,thick, out=35,in=145,looseness=0.75]%
% ([xshift=0.2cm, yshift=0.4cm]l1.north east) to%
% node[yshift = 0.2cm, above]{$u^{1}_{i,j} = f^0\left(W^0_{i,k}u^0_{i,k} + \theta^{n_h-1}_i\right)$}%
% ([xshift=-0.2cm, yshift=0.4cm]l2.north west);



% \draw [->,thick, out=0,in=120,looseness=0.75]%
% ([xshift=0.2cm, yshift=0.4cm]lnh. east) to%
% node[yshift = 0.2cm, right]{$y_i = g\left(W^{l_{n_h}}_{0,k}u^{l_{n_h}}_{i,k} + \theta^{l_{n_h}} \right)$}%
% \matrix (B) [matrix of math nodes,%
%              left delimiter  = {[},%
%              right delimiter = {]}] at (5, 0)
% {%
%   u^{l+1}_{0,0} & \ldots & u^{l+1}_{0,m}  \\
%   u^{l+1}_{1,0} & \ldots & u^{l+1}_{1,m}  \\
%    \vdots &        & \vdots   \\
%   u^{l+1}_{n,0} & \ldots & u^{l+1}_{n,m}  \\
% };

% \matrix (dA) [matrix of math nodes,%
%              left delimiter  = {[},%
%              right delimiter = {]}] at (0, -4)
% {%
%   \frac{dy}{dW^l_{0,0}} & \ldots & \frac{dy}{dW^l_{0,m}}  \\
%   \frac{dy}{dW^l_{1,0}} & \ldots & \frac{dy}{dW^l_{1,m}}  \\
%    \vdots &        & \vdots   \\
%   \frac{dy}{dW^l_{n,0}} & \ldots & \frac{dy}{dW^l_{n,m}}  \\
% };

% \matrix (dB) [matrix of math nodes,%
%              left delimiter  = {[},%
%              right delimiter = {]}] at (5, -4)
% {%
%   \frac{dy}{dW^{l+1}_{0,0}} & \ldots & \frac{dy}{dW^{l+1}_{0,m}}  \\
%   \frac{dy}{dW^{l+1}_{1,0}} & \ldots & \frac{dy}{dW^{l+1}_{1,m}}  \\
%    \vdots &        & \vdots   \\
%   \frac{dy}{dW^{l+1}_{n,0}} & \ldots & \frac{dy}{dW^{l+1}_{n,m}}  \\
% };

% \matrix (C) [matrix of math nodes,%
%              left delimiter  = {[},%
%              right delimiter = {]}] at (10, 0)
% {%
%   u^2_{0,0} & \ldots & u^2_{0,m}  \\
%   u^2_{1,0} & \ldots & u^2_{1,m}  \\
%    \vdots &        & \vdots   \\
%   u^2_{n,0} & \ldots & u^2_{n,m}  \\
% };


    % \draw [->,thick, out=-135,in=-45,looseness=0.75]%
    % ([xshift=-0.4cm, yshift=-0.4cm]dB.south west) to
    % node[yshift = 0.2cm, below]{$\frac{dy}{du^{l+1}_{i,j}} = f^{l+1}_i\left( W^{l+1}_{ik} u^l_{j,k} + \theta^{1+1}_i \right)$}%
    % ([xshift=0.4cm, yshift=-0.4cm]dA.south east);

  % l layer
% \draw node[draw=black, circle, minwidth=2cm] (A) at (1, 0) {$u^{l}_{i,j}$};
% \draw node[draw=black, circle, minwidth=2cm] (dAu) at (1, -3) {$\frac{dy}{du^{l}_{i,j}$};
% \draw node[draw=black, circle, minwidth=2cm] (dAw) at (1, -5) {$\frac{dy}{dW^{l}_{i,j}$};
% \draw node[draw=black, circle, minwidth=2cm] (dAt) at (1, -7) {$\frac{dy}{d\theta^l_{i}$};

  % l+1 layer
% \draw node[draw=black, circle, minwidth=2cm] (B) at (5, 0) {$u^{l+1}_{i,j}$};
% \draw node[draw=black, circle, minwidth=2cm] (dBu) at (5, -3) {$\frac{dy}{du^{l+1}_{i,j}$};
% \draw node[draw=black, circle, minwidth=2cm] (dBw) at (5, -5) {$\frac{dy}{dW^{l+1}_{i,j}$};
% \draw node[draw=black, circle, minwidth=2cm] (dBt) at (5, -7) {$\frac{dy}{d\theta^{l+1}_{i}$};


% \draw [->,thick, out=135,in=45,looseness=0.75]%
% ([xshift=0.4cm, yshift=+0.4cm]dBu.north west) to%
% node[above]{$\frac{dy}{du^{l}_{i,j}} = W^l_{i,k} \frac{df}{dx}\left(W^l_{i,m}u^l_{i,m}+\theta^l_i\right) \frac{dy}{du^{l+1}_{i,j}}$}
% ([xshift=-0.4cm, yshift=+0.4cm]dAu.north east);
\end{tikzpicture}
%%% Local Variables:
%%% mode: latex
%%% TeX-master: t
%%% End:
%
}
\caption{Forward propagation of the neuron values through the neural network.
  Repeated indices indicate summation over those indices.}
\label{fig:forward}
\end{figure}

The algorithm is given in pseudocode below. The computation of the neuron
 activations of each layer thus requires
\begin{itemize}
  \item computation of the tensor product $W^l_{j,k}u^l_{i,k}$,
  \item addition of the bias vectors $\theta^l_k$ along the second dimension
    of the tensor,
  \item application of the activation function $f^l$ to this tensor.
\end{itemize}


\begin{algorithm}
  \caption{Forward Propagation} \label{alg:gn}
\begin{algorithmic}[1]
  \State $u^1_{i,j} \leftarrow f^1 \left (W^1_{j,k}x_{i,k} + \theta^1_j \right )$
  \For {$l = 1, \ldots, n_h$}
  \State $u^l_{i,j} \leftarrow f^l \left (W^l_{j,k}u^l_{i,k} + \theta^l_j \right )$
  \EndFor
  \State $\text{obj} \leftarrow J \left (\mathbf{u}^l, \mathbf{y}, \mathbf{W} \right)$
\end{algorithmic}
\end{algorithm}

\subsection{Backward Propagation}

The gradients of the objective function with respect to the weights of each layer
are computed by repeated application of the chain rule of calculus. Starting from
the gradients of the objective function with respect to the activations $u^{n_h}_{i,j}$
 of the output layer $\frac{dJ}{du^{n_h}_{i,j}}$ the gradients can be computed from
\begin{align}
  \frac{dJ}{dW^l_{i,j}}      &= u^{l-1}_{m,j} (f^l)'_{m,i} \frac{dJ}{du^l_{m,i}} + R(W^l_{i,j}) \label{eq:bp1} \\
  \frac{dJ}{d\theta^l_{i}} &= (f^l)'_{m,i} \frac{dJ}{du^l_{m,i}} \\
  \frac{dJ}{du^{l-1}_{i,j}}  &= W^l_{n,j}(f^l)'_{i,n} \frac{dJ}{du^l_{i,n}} \label{eq:bp2}
\end{align}

Here, $R(W^n_{i,j})$ is an additional contribution to the gradient from
potential regularization terms in the objective function. The term
$(f^l)'_{m,i}$ is used to denote the first derivative of the
activation function evaluated at $W^l_{i,k}u^{l-1}_{m,k} +
\theta_i$. Note that the computations in equation~\eqref{eq:bp1, eq:bp2} can be
implemented using element-wise as well as standard and transposed matrix-matrix
 multiplication.

\begin{figure}[!h]
\centering
\resizebox{\linewidth}{!}{%
  %%% Local Variables:
%%% mode: latex
%%% TeX-master: t
%%% End:
\begin{tikzpicture}[>=latex]
% les matrices
% \matrix (X) [matrix of math nodes,%
%              left delimiter  = {[},%
%              right delimiter = {]}] at (0, -1)
% {%
%   x_{0,0} & \ldots & x_{0,m}  \\
%   x_{1,0} & \ldots & x_{1,m}  \\
%    \vdots &        & \vdots   \\
%   x_{n,0} & \ldots & x_{n,m}  \\
% };

%l1
\draw[fill=black!20, draw=none] (3,2) rectangle (5,-4) {};
\draw node[fill=blue!50, circle] (l1) at (4, 1) {$\frac{dJ}{du^{0}_{i,j}$};
\draw node[fill=green!50, circle] (W1) at (4, -1) {$\frac{dJ}{dW^{0}_{i,j}}$};
\draw node[fill=green!50, circle] (t1) at (4, -2.5) {$\frac{dJ}{d\theta^{0}_{i}}$};

% \draw [->,thick]%
% ([xshift=0.4cm]X. east) to%
% ([xshift=-0.2cm]l1.south west);

\draw [->,thick]%
([yshift=-0.1cm]l1.south) to%
node[yshift = 0.2cm, above]{}%
([yshift=0.1cm]W1.north);

\draw [->, thick, out=225, in=135]%
([yshift=-0.1cm]l1.south west) to%
node[yshift = 0.2cm, above]{}%
([yshift=0.1cm]t1.north west);


%l2
\draw[fill=black!20, draw=none] (7,2) rectangle (9,-4) {};
\draw node[fill=blue!50, circle] (l2) at (8, 1) {$\frac{dJ}{du^{1}_{i,j}}$};
\draw node[fill=green!50, circle] (W2) at (8, -1) {$\frac{dJ}{dW^{0}_{i,j}}$};
\draw node[fill=green!50, circle] (t2) at (8, -2.5) {$\frac{dJ}{d\theta^{0}_{i}}$};

\draw [->,thick]%
([xshift=-0.2cm]l2.west) to
([xshift=+0.2cm]l1. east);

\draw [->,thick]%
([yshift=0.-0.1cm]l2.south) to
node[yshift = 0.2cm, above]{}%
([yshift=0.1cm]W2.north);

\draw [->, thick, out=225, in=135]%
([yshift=-0.1cm]l2.south west) to
([yshift=+0.1cm]t2.north west);

%ld
\draw node[draw=none] (ld) at (10.5, 1) {$\ldots$};
\draw [->,thick]%
([xshift=-0.2cm]ld.west) to
([xshift=0.2cm]l2.east);

%lnh
%\draw node at (13,-5) {$u^{n_h}_{i,j} = f^1 \left (W^{n_h}_{j,k}u^{n_h}_{i,k} + \theta^{n_h}_j \right )$};
%\draw node at (13,-5) {$\frac{dJ}{du^{n_h}_{i,j}} = f^1 \left (W^{n_h}_{j,k}u^{n_h}_{i,k} + \theta^{n_h}_j \right )$};
\draw node at (14,-6.05) {%
$\begin{aligned}
  & \\ \\
  \frac{dJ}{dW^{n_h}_{i,j}} & = u^{n_h}_{m,j} \left ( f^{n_h} \right )'_{m,i}\frac{dJ}{du^{n_h}_{m,i}} \\
\frac{dJ}{d\theta^{n_h}_{i}} & = \left (f^{n_h} \right )'_{m,i} \frac{dJ}{du^{n_h}_{m,i}} \\
\end{aligned}$};

\draw[fill=black!20, draw=none] (12,2) rectangle (14,-4) {};
\draw node[fill=blue!50, circle] (lnh) at (13, 1) {$\frac{dJ}{du^{n_h}_{i,j}}$};
\draw node[fill=green!50, circle] (Wnh) at (13, -1) {$\frac{dJ}{dW^{n_h}_{j,j}}$};
\draw node[fill=green!50, circle] (tnh) at (13, -2.5) {$\frac{dJ}{d\theta^{n_h}_{j}}$};

\draw node at (9,-6) {%
$\begin{aligned}
  \frac{dJ}{du^{n}_{i,j}} & = W^{l}_{n,j} \left ( f^{n} \right )'_{i,n}\frac{dJ}{du^{n+1}_{i,n}} \\
  \frac{dJ}{dW^{n}_{i,j}} & = u^{n}_{m,j} \left ( f^{n} \right )'_{m,i}\frac{dJ}{du^{n}_{m,i}} \\
\frac{dJ}{d\theta^{n}_{i}} & = \left (f^{n} \right )'_{m,i} \frac{dJ}{du^{n}_{m,i}} \\
\end{aligned}$};

\draw node at (3,-6) {%
$\begin{aligned}
  \frac{dJ}{du^{n}_{i,j}} & = W^{l}_{n,j} \left ( f^{n} \right )'_{i,n}\frac{dJ}{du^{n+1}_{i,n}} \\
  \frac{dJ}{dW^{n}_{i,j}} & = u^{n}_{m,j} \left ( f^{n} \right )'_{m,i}\frac{dJ}{du^{n}_{m,i}} \\
\frac{dJ}{d\theta^{n}_{i}} & = \left (f^{n} \right )'_{m,i} \frac{dJ}{du^{n}_{m,i}} \\
\end{aligned}$};

\draw [->,thick]%
([xshift=-0.2cm]lnh.west) to
([xshift=0.2cm]ld.east);

\draw [->,thick]%
([yshift=-0.1cm]lnh.south) to
([yshift=0.1cm]Wnh.north);

\draw [->, thick, out=225, in=135]%
([yshift=-0.1cm]lnh.south west) to
(tnh.north west);

% \draw [->, thick]%
% ([xshift=0.2cm]lnh.south east) to%
% ([xshift=-0.4cm]Y.west);

% \matrix (Y) [matrix of math nodes,%
%              left delimiter  = {[},%
%              right delimiter = {]}] at (16, -5)
% {%
% y_{0} \\
% y_{1} \\
% \vdots\\
% y_{n} \\
% };

\draw node[fill=red!50, circle] (J) at (17, -1) {$J(\mathbf{y},\mathbf{\hat{y}})$};
% Error function

\draw [->, thick]%
([xshift=-0.2cm]J.north west) to%
([xshift=+0.2cm]lnh.south east);


% \draw [->,thick, out=35,in=145,looseness=0.75]%
% ([xshift=0.2cm, yshift=0.4cm]l1.north east) to%
% node[yshift = 0.2cm, above]{$u^{1}_{i,j} = f^0\left(W^0_{i,k}u^0_{i,k} + \theta^{n_h-1}_i\right)$}%
% ([xshift=-0.2cm, yshift=0.4cm]l2.north west);



% \draw [->,thick, out=0,in=120,looseness=0.75]%
% ([xshift=0.2cm, yshift=0.4cm]lnh. east) to%
% node[yshift = 0.2cm, right]{$y_i = g\left(W^{l_{n_h}}_{0,k}u^{l_{n_h}}_{i,k} + \theta^{l_{n_h}} \right)$}%
% \matrix (B) [matrix of math nodes,%
%              left delimiter  = {[},%
%              right delimiter = {]}] at (5, 0)
% {%
%   u^{l+1}_{0,0} & \ldots & u^{l+1}_{0,m}  \\
%   u^{l+1}_{1,0} & \ldots & u^{l+1}_{1,m}  \\
%    \vdots &        & \vdots   \\
%   u^{l+1}_{n,0} & \ldots & u^{l+1}_{n,m}  \\
% };

% \matrix (dA) [matrix of math nodes,%
%              left delimiter  = {[},%
%              right delimiter = {]}] at (0, -4)
% {%
%   \frac{dy}{dW^l_{0,0}} & \ldots & \frac{dy}{dW^l_{0,m}}  \\
%   \frac{dy}{dW^l_{1,0}} & \ldots & \frac{dy}{dW^l_{1,m}}  \\
%    \vdots &        & \vdots   \\
%   \frac{dy}{dW^l_{n,0}} & \ldots & \frac{dy}{dW^l_{n,m}}  \\
% };

% \matrix (dB) [matrix of math nodes,%
%              left delimiter  = {[},%
%              right delimiter = {]}] at (5, -4)
% {%
%   \frac{dy}{dW^{l+1}_{0,0}} & \ldots & \frac{dy}{dW^{l+1}_{0,m}}  \\
%   \frac{dy}{dW^{l+1}_{1,0}} & \ldots & \frac{dy}{dW^{l+1}_{1,m}}  \\
%    \vdots &        & \vdots   \\
%   \frac{dy}{dW^{l+1}_{n,0}} & \ldots & \frac{dy}{dW^{l+1}_{n,m}}  \\
% };

% \matrix (C) [matrix of math nodes,%
%              left delimiter  = {[},%
%              right delimiter = {]}] at (10, 0)
% {%
%   u^2_{0,0} & \ldots & u^2_{0,m}  \\
%   u^2_{1,0} & \ldots & u^2_{1,m}  \\
%    \vdots &        & \vdots   \\
%   u^2_{n,0} & \ldots & u^2_{n,m}  \\
% };


    % \draw [->,thick, out=-135,in=-45,looseness=0.75]%
    % ([xshift=-0.4cm, yshift=-0.4cm]dB.south west) to
    % node[yshift = 0.2cm, below]{$\frac{dy}{du^{l+1}_{i,j}} = f^{l+1}_i\left( W^{l+1}_{ik} u^l_{j,k} + \theta^{1+1}_i \right)$}%
    % ([xshift=0.4cm, yshift=-0.4cm]dA.south east);

  % l layer
% \draw node[draw=black, circle, minwidth=2cm] (A) at (1, 0) {$u^{l}_{i,j}$};
% \draw node[draw=black, circle, minwidth=2cm] (dAu) at (1, -3) {$\frac{dy}{du^{l}_{i,j}$};
% \draw node[draw=black, circle, minwidth=2cm] (dAw) at (1, -5) {$\frac{dy}{dW^{l}_{i,j}$};
% \draw node[draw=black, circle, minwidth=2cm] (dAt) at (1, -7) {$\frac{dy}{d\theta^l_{i}$};

  % l+1 layer
% \draw node[draw=black, circle, minwidth=2cm] (B) at (5, 0) {$u^{l+1}_{i,j}$};
% \draw node[draw=black, circle, minwidth=2cm] (dBu) at (5, -3) {$\frac{dy}{du^{l+1}_{i,j}$};
% \draw node[draw=black, circle, minwidth=2cm] (dBw) at (5, -5) {$\frac{dy}{dW^{l+1}_{i,j}$};
% \draw node[draw=black, circle, minwidth=2cm] (dBt) at (5, -7) {$\frac{dy}{d\theta^{l+1}_{i}$};


% \draw [->,thick, out=135,in=45,looseness=0.75]%
% ([xshift=0.4cm, yshift=+0.4cm]dBu.north west) to%
% node[above]{$\frac{dy}{du^{l}_{i,j}} = W^l_{i,k} \frac{df}{dx}\left(W^l_{i,m}u^l_{i,m}+\theta^l_i\right) \frac{dy}{du^{l+1}_{i,j}}$}
% ([xshift=-0.4cm, yshift=+0.4cm]dAu.north east);
\end{tikzpicture}
%%% Local Variables:
%%% mode: latex
%%% TeX-master: t
%%% End:
%
}
\caption{Backward propagation of the gradients through the neural network.
  Repeated indices indicate summation over those indices.}
\label{fig:backward}
\end{figure}

\section{Low-level Interface}

In this section the low-level interface is presented, which will
separate the compute intensive, mathematical operations from the
general coordination of the training.

\subsection{Forward Propagation}

We split the forward propagation of the activations through a given layer into
two steps. In the first step, the linear activations are computed using

\begin{align}
  \mathbf{t}^l &= \mathbf{u}^{l-1} \mathbf{W}^T + \boldsymbol{\theta}^l
\end{align}

These operations are implemented by the \code{multiply_transpose} and the
\code{add_row_wise} functions in the low-level interface.

\begin{minted}{c++}
void multiply_transpose( &output,
                        const MatrixType &input,
                        const MatrixType &weights)

void add_row_wise(MatrixType &output,
                  const MatrixType &biases)

\end{minted}

In the second step, the non-linear activation function of the layer are applied
 to the temporary result $\mathbf{t}^l$ yielding the activations
 $\boldsymbol{u}^{l}$ of the current layer.

\begin{align}
  \mathbf{u}^l &= f \left ( \mathbf{u}^{l-1} \mathbf{W}^T + \boldsymbol{\theta}^l
                 \right )
\end{align}

These operations are implemented using the \code{evaluate} function described in
Section~\ref{ss:activations}.

\subsection{Backward Propagation}

The backward propagation is implemented by a single method in the
low-level interface. For a given layer $l$, this function takes the
gradients of the objective function with respect to the activations of
the $l+1$ (forward direction) layer, the weights of the current layer
and the activations of the $l-1$ layer (backward direction).  It
also takes as input the first derivatives of the activation functions, which
should ideally be computed during the forward propagation phase. The
\code{backward} method computes the gradient of the objective function with
 respect to the activation energies of the previous layer, the weights of
 the layer as well as the biases.

Note that the formulas for backpropagation can be implemented using
element-wise matrix multiplication as well as standard and transposed
matrix-matrix multiplication.

\begin{minted}{c++}
template<typename RealType>
    void backward(MatrixType & activation_gradients_backward,
                MatrixType & weight_gradients,
                MatrixType & bias_gradients,
                MatrixType & df,
                const MatrixType & activation_gradients,
                const MatrixType & weights,
                const MatrixType & activations_backward,
                Regularization r)
\end{minted}

\subsection{Activation Functions}\label{ss:activations}

The activation functions are represented by the \code{ActivationFunction} enum
class.

\begin{minted}{c++}
        /*! Enum that represents layer activation functions. */
        enum class ActivationFunction
        {
            IDENTITY = 'I',
            RELU     = 'R'
        };
\end{minted}

The evaluation of the activation functions is performed using the
\code{evaluate} function template which forwards the call to the
actual evaluation function corresponding to the given type of the
activation function. An architecture-specific implementation simply
overloads those evaluation function for the architecture-specific
matrix data type. As an example the signature for the evaluation of
the ReLU function is given below:

\begin{minted}{c++}
// Apply the ReLU functions to the elements in A.
inline void relu(MatrixType &A);
\end{minted}

In addition to the evaluation of the activation functions, also a function that
 computes the first order derivatives of the activation functions must be provided.
 The signature for the computation of the first order derivatives of the ReLU
 function is also given below.

\begin{minted}{c++}
// For each element in A evaluate the first derivative of the ReLU function
// and write the result into B.
inline void relu_derivative(MatrixType &B, const MatrixType &A);
\end{minted}

\subsection{Loss Functions}

Similar to the activation functions, the loss functions are also represented by an
enum class.

\begin{minted}{c++}
enum class LossFunction
{
    MEANSQUAREDERROR = 'R'
};
\end{minted}

The evaluation of the loss functions is implemented by the overloaded
 \code{evaluate} function, which computes the loss from a given activation
matrix of the output layer of the network and the matrix $\mathbf{Y}$ containing
the true predictions. Similar to the implementation of activation functions,
the generic \code{evaluate} function forwards the call to a given
 device-pecific evaluation function, that is overloaded with the device-specific
 matrix type.

\begin{minted}{c++}
template<typename MatrixType>
    inline double evaluate(LossFunction f,
                    const MatrixType & Y,
                    const MatrixType & output)
\end{minted}

In addition to that, we need to be able to compute the gradient of the loss
function with respect to the activations of the output layer. This is
implemented by the \code{evaluate_gradient} function, which also just forwards
the call for a given loss function to the corresponding device-specific function.

\begin{minted}{c++}
        template<typename MatrixType>
            inline void evaluate_gradient(MatrixType & dY,
                                   LossFunction f,
                                   const MatrixType &Y,
                                   const MatrixType &output)
\end{minted}

Currently only the mean squared error function is implemented as a loss functions
the function signature for the call to the device specific method is given below:

\begin{minted}{c++}
template<typename RealType>
inline RealType mean_squared_error(const TMatrixT<RealType> &Y,
                                   const TMatrixT<RealType> &output)
\end{minted}

\subsection{Regularization}

For the treatment of regularization, we proceed in a similar way as above. The type
of the regularization is represented by an enum class

\begin{minted}{c++}
/*! Enum representing the regularization type applied for a given layer */
enum class Regularization
{
    NONE = '0',
    L1   = '1',
    L2   = '2'
};
\end{minted}

The generic \code{regularization} function resolves the type of the regularization
and forwards the call to the corresponding device specific method.

\begin{minted}{c++}
template<typename MatrixType>
    auto regularization(const MatrixType & A,
                        Regularization R)
    -> decltype(l1_regularization(A));
\end{minted}

In addition to that, we need to add the gradient of the regularization to the
gradients of the prediction loss function. This is implemented by the
\code{add_regularization_gradient} method.

\begin{minted}{c++}
template<typename MatrixType>
    void add_regularization_gradient(MatrixType &A,
                                     const MatrixType &W,
                                     Regularization R)
\end{minted}

The type signatures of the device specific functions that compute the L2
regularization for a given weight matrix and add the gradients of the L2
regularization term with respect to a given matrix to another matrix are
given below.

\begin{minted}{c++}
inline RealType l2_regularization(const MatrixType & W)
inline void     add_l2_regularization_gradient(MatrixType & A,
                                               MatrixType & W)
\end{minted}

And similarly for L1 regularization:

\begin{minted}{c++}
inline RealType l1_regularization(const MatrixType & W)
inline void     add_l1_regularization_gradient(MatrixType & A,
                                               const MatrixType & W)
\end{minted}

\subsection{Initialization}

The initialization of the layers is treated in a similar faschion as the
activation and output functions as well as the regularization. An enum class
specifies which type of initialization should be performed and the and a generic
function the forward the call to the initialize function to the device-specific
implementation of the desired initialization method.

\begin{minted}{c++}
/* Enum represnting the initialization method used for this layer. */
enum class InitializationMethod
{
    GAUSS    = 'G',
    UNIFORM  = 'U',
    IDENTITY = 'I'
};
\end{minted}

The function signatures for the device-specific initialization methods are given
below:

\begin{minted}{c++}
inline void initialize_gauss(MatrixType & A);
inline void initialize_uniform(MatrixType & A);
inline void initialize_identity(MatrixType & A);
\end{minted}

% \section{Minimization}

% The current DNN implementation provides a steepest descent minimizer with
% momentum. A minimizer step thus only involves adding a scaled version of a
% of a matrix or a vector to another matrix or vector.

% \begin{minted}{c++}
%   MatrixType & MatrixType opertor*=(const MatrixType &A, double c = 1.0);
% \end{minted}



\end{document}
%%% Local Variables:
%%% mode: latex
%%% TeX-master: t
%%% End:
